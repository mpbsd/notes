\section{The elements of the theory}\label{sec:the_elements_of_the_theory}

\begin{theorem}\label{thm:distinct-field-automorphisms-are-linearly-independent}
	If $K$ is a field and if $\sigma_{1},\ldots,\sigma_{n}$ are distinct automorphisms of $K$ then it is impossible to find in $K$ elements $a_{1},\ldots,a_{n}$ not all equal to $0$ such that $a_{1}\sigma(u)+\cdots+a_{n}\sigma(u)=0$ for all ${u}\in{K}$.
\end{theorem}

\begin{definition}\label{def:fixed-field}
	If $G$ is a group of automorphisms of $K$, then the fixed field of $G$ is the set of all elements ${a}\in{K}$ such that $\sigma(a)=a$ for all ${\sigma}\in{G}$.
\end{definition}

\begin{lemma}\label{lem:the-fixed-field-is-in-fact-a-subfield}
	The fixed field of $G$ is a subfield of $K$.
\end{lemma}

\begin{definition}\label{def:automorphism-group-relative-to-a-subfield}
	Let $K$ be a field and let $F$ be a subfield of $K$. Then the group of automorphisms of $K$ relative to $F$, written as $G(K,F)$, is the set of all automorphisms of $K$ leaving every element of $F$ point wise fixed; that is, the automorphism $\sigma$ of $K$ is in $G(K,F)$ if and only if $\sigma(\alpha)=\alpha$ for every ${\alpha}\in{K}$.
\end{definition}

\begin{theorem}\label{thm:finite-extensions-make-a-finite-relative-automorphism-group}
	If $K$ is a finite extension of $F$ then $G(K,F)$ is a finite group and its order is not greater than $[K:F]$.
\end{theorem}

\begin{theorem}\label{thm:symmetric-rational-functions}
	Let $F$ be a field and let $F(x_{1},\ldots,x_{n})$ be the field of rational functions in $x_{1},\ldots,x_{n}$ over $F$. Let $S$ be the field of symmetric rational functions. Then:
	\begin{enumerate}
		\item $[F(x_{1},\ldots,x_{n}):S]=n!$;
		\item $G(F(x_{1},\ldots,x_{n}),S)=S_{n}$, the symmetric group of degree $n$;
		\item If $a_{1},\ldots,a_{n}$ are the elementary symmetric functions in $x_{1},\ldots,x_{n}$, then $S=F(a_{1},\ldots,a_{n})$;
		\item $F(x_{1},\ldots,x_{n})$ is the splitting field over $F(a_{1},\ldots,a_{n})=S$ of the polynomial $t^{n}-a_{1}t^{n-1}+a_{2}t^{n-2}+\cdots+(-1)^{n}a_{n}$.
	\end{enumerate}
\end{theorem}

\begin{definition}\label{def:normal-extension}
	$K$ is a normal extension of $F$ if $K$ is a finite extension of $F$ such that $F$ is the fixed field of $G(K,F)$.
\end{definition}

\begin{theorem}\label{thm:on-the-subfield-subgroup-correspondence}
	Let $K$ be a normal extension of $F$ and let $H$ be a subgroup of $G(K,F)$. Let
	\[
		K_{H}=\left\{{x}\in{K}:\sigma(x)=x\text{ for all }{\sigma}\in{H}\right\}
	\]
	be the fixed field of $H$. Then:
	\begin{enumerate}
		\item $[K:K_{H}]=\order{H}$
		\item $H=G(K,K_{H})$
	\end{enumerate}
\end{theorem}

\begin{theorem}\label{thm:a-characterization-of-normality}
	$K$ is a normal extension of $F$ if and only if $K$ is the splitting field of some polynomial over $F$.
\end{theorem}

\begin{lemma}\label{lem:the-action-of-the-galois-group-of-a-normal-extension-is-transitive}
	Let $K$ be the splitting field of $f(x)$ in $F[x]$ and let $p(x)$ be an irreducible factor of $f(x)$ in $F[x]$. If the roots of $p(x)$ are $\alpha_{1},\ldots,\alpha_{r}$, then for each ${i}\in\{1,\ldots,r\}$ there is an automorphism ${\sigma_{i}}\in{G(K,F)}$ such that $\sigma_{i}(\alpha_{1})=\alpha_{i}$.
\end{lemma}

\begin{definition}\label{def:galois-group}
	Let $f(x)$ be a polynomial in $F[x]$ and let $K$ be its splitting field over $F$. The Galois group of $f(x)$ is the group $G(K,F)$ of all the automorphisms of $K$ leaving every element of $F$ point wise fixed.
\end{definition}

\begin{lemma}\label{lem:another-characterization-of-normality}
	Let $K$ be the splitting field over $F$ of some polynomial ${f(x)}\in{F[x]}$. Then, an extension $T$ of $F$ in $K$ is a normal extension of $F$ if and only if $\sigma(T)\subset{T}$ for every ${\sigma}\in{G(K,F)}$.
\end{lemma}

\begin{proof}
	Let's write $G(K,F)=\{\sigma_{1}=id,\sigma_{2},\ldots,\sigma_{n}\}$, where $n=\order{G(K,F)}=[K:F]$. Then, assume that $T$ is any extension of $F$ in $K$ such that $\sigma(T)\subset{T}$ for every ${\sigma}\in{G(K,F)}$. We want to show that $T$ is the splitting field of some polynomial with coefficients in $F$. Since $[T:F]$ is finite, in fact, $[T:F]\leqslant{[K:F]}$, we get that $T=F(a)$ for some ${a}\in{T}$. Then, the polynomial
	\begin{align*}
		p(t)
		 & =
		\prod\left\{t-\sigma(a):{\sigma}\in{G(K,F)}\right\}
		\\
		 & =
		(t-a)(t-\sigma_{2}(a))\cdots(t-\sigma_{n}(a))
		\\
		 & =
		t^{n}-\alpha_{1}t^{n-1}+\alpha_{2}t^{n-2}+\cdots+(-1)^{n}\alpha_{n}
	\end{align*}
	is a polynomial in $T[t]$, where $\alpha_{1},\alpha_{2},\ldots,\alpha_{n}$ are the elementary symmetric functions in $\sigma_{1}(a)=a,\sigma_{2}(a),\ldots,\sigma_{n}(a)$. Recall that the $\sigma_{i}(a)$ must be distinct from one another because $a$ is the root of an irreducible polynomial over $F$, namely, the minimal polynomial of $a$ over $F$, and the $\sigma_{i}$ belong to $G(K,F)$. Now, $\sigma(\alpha_{i})=\alpha_{i}$ for every ${\sigma}\in{G(K,F)}$ and as such $\alpha_{i}$ belongs to $K_{G(K,F)}=F$ for every ${i}\in{\{1,\ldots,n\}}$. Thus, $p(t)$ is in reality a polynomial in $F[t]$. Finally, if $S$ is any extension of $F$ that contains a complete system of roots of $p(t)$, then it also contains $a$. As a consequence, we get that $T=F(a)\subset{S}$ and $[T:F]\leqslant{[S:F]}$. Therefore, $T=F(a)$ is the splitting field of a polynomial with coefficients in $F$.

	Conversely, suppose that $T$ is a normal extension of $F$ in $K$. We want to show that ${\sigma(T)}\subset{T}$ for every ${\sigma}\in{G(K,F)}$. Let's write $G(T,F)=\{\sigma_{1}=id,\sigma_{2},\ldots,\sigma_{m}\}$, where $m=\order{G(T,F)}=[T:F]$. Once again, we have $T=F(a)$ for some ${a}\in{T}$. Now, let
	\begin{align*}
		q(t)
		 & =
		\prod\left\{t-\sigma(a):{\sigma}\in{G(T,F)}\right\}
		\\
		 & =
		(t-a)(t-\sigma_{2}(a))\cdots(t-\sigma_{m}(a))
		\\
		 & =
		t^{m}-\beta_{1}t^{m-1}+\beta_{2}t^{m-2}+\cdots+(-1)^{m}\beta_{m}
	\end{align*}
	where $\beta_{1},\beta_{2},\ldots,\beta_{m}$ are the elementary symmetric functions in $\sigma_{1}(a)=a,\sigma_{2}(a),\ldots,\sigma_{m}(a)$. As before, the $\sigma_{i}(a)$ are distinct from one another because $a$ is the root of an irreducible polynomial over $F$, that being the minimal polynomial of $a$ over $F$, and the $\sigma_{i}$ belong to $G(T,F)$. Here too, the equality $\sigma(\beta_{i})=\beta_{i}$ holds for every ${\sigma}\in{G(T,F)}$, which implies that the $\beta_{i}$ belong to $T_{G(T,F)}=F$ for every ${i}\in{\{1,\ldots,m\}}$. Thus, $q(t)$ is a monic polynomial in $F[t]$ of degree $m=[F(a):F]$ having $a$ as one of its roots. Therefore, $q(t)$ must be the minimal polynomial of $a$ over $F$. Finally, for any given automorphism $\sigma$ in $G(K,F)$, the element $\sigma(a)$ is again a root of $q(t)$ and as such it belongs to the set ${\{\sigma(a):{\sigma}\in{G(T,F)}\}}=\{\sigma_{1}(a)=a,\sigma_{2}(a),\ldots,\sigma_{m}(a)\}$ which is itself a subset of $T$. Since $T$ is generated by $a$ over $F$, this implies that ${\sigma(T)}\subset{T}$ for every ${\sigma}\in{G(K,F)}$.
\end{proof}

\begin{theorem}\label{thm:galois-correspondence}
	Let $K$ be the splitting field over $F$ of some polynomial ${f(x)}$ in $F[x]$ and let $G(K,F)$ be its Galois group. For any subfield $T$ of $K$ containing $F$, let
	\[
		G(K,T)=\left\{{\sigma}\in{G(K,F)}:\sigma(t)=t\text{ for all }{t}\in{T}\right\}
	\]
	and, for any subgroup $H$ of $G(K,F)$, let
	\[
		K_{H}=\left\{{x}\in{K}:\sigma(x)=x\text{ for every }{\sigma}\in{H}\right\}.
	\]
	Then, the association ${T}\mapsto{G(K,T)}$ sets up a one-to-one correspondence between the set of subfields of $K$ containing $F$ onto the set of subgroups of $G(K,F)$ in such a way that:
	\begin{enumerate}
		\item $T=K_{G(K,T)}$
		\item $H=G(K,K_{H})$
		\item $[K:T]=\order{G(K,T)}$ and $[T:F]=[G(K,F):G(K,T)]$
		\item $T$ is a normal extension of $F$ if and only if $G(K,T)$ is a normal subgroup of $G(K,F)$
		\item When $T$ is a normal extension of $F$, then $G(T,F)$ is isomorphic to $G(K,F)/G(K,T)$.
	\end{enumerate}
\end{theorem}

\section{Problems}\label{sec:problems}

These exercises might be found in~\cite{herstein_topics_2010}.

\begin{problem}\label{problem:6}
Prove directly that any automorphism of $K$ must leave every rational number fixed.
\end{problem}

\begin{problem}\label{problem:17}
For each of the following statements, do what is being requested.
\begin{enumerate}
	\item Find the splitting field $K$ of $x^{3}-2$ over $\mathbb{Q}$;
	\item Prove that the Galois group of $x^{3}-2$ over $\mathbb{Q}$ is isomorphic to $S_{3}$, the symmetric group of degree $3$;
	\item For each ${H}\leqslant{S_{3}}$ find $K_{H}$ and manually check the correspondence given in Theorem~\ref{thm:galois-correspondence}.
	\item Find a normal extension extension in $K$ of degree $2$ over $\mathbb{Q}$.
\end{enumerate}
\end{problem}

\begin{problem}\label{problem:18}
If the field $F$ contains a primitive nth root of unity, prove that the Galois group of $x^{n}-a$ is abelian for every ${a}\in{F}$.
\end{problem}

% \begin{solution}
% 	Let ${\omega}\in{F}$ be a primitive nth root of unity. For any ${a}\in{F}$, let $a=re^{i\theta}$ be the polar form of $a$ and take $b=\sqrt[n]{r}e^{i\frac{\theta}{n}}$. Then, $F(b)$ contains all roots $b,b\omega,b\omega^{2},\ldots,b\omega^{n-1}$ of $x^{n}-a$ and any field containing both $F$ and the roots of $x^{n}-a$ must also contain $F(b)$. Thus, $F(b)$ is the splitting field of $x^{n}-a$ over $F$. Now, for any elements $\sigma,\tau$ in $G(F(b),F)$, let's write $\sigma(b)=b\omega^{i}$ and $\tau(b)=b\omega^{j}$. Then, we get that
% 	\[
% 		\sigma\tau(b)=\sigma(b\omega^{j})=\sigma(b)\omega^{j}=\beta\omega^{i}\omega^{j}=b\omega^{i+j}=
% 		b\omega^{j+i}=b\omega^{j}\omega^{i}=\tau(b)\omega^{i}=\tau(b\omega^{i})=\tau\sigma(b),
% 	\]
% 	and because $F(b)$ is generated by $b$ over $F$, this even implies that $\sigma\tau=\tau\sigma$ on all of $F(b)$.
% \end{solution}

\begin{problem}\label{problem:13}
Prove each of the following assertions.
\begin{enumerate}
	\item There are $\phi(n)$ primitive nth roots of unity where $\phi(n)$ is the Euler $\phi$-function.
	\item If $\omega$ is a primitive nth root of unity prove that $\mathbb{Q}(\omega)$ is the splitting field of $x^{n}-1$ over $\mathbb{Q}$ (and so is a normal extension of $\mathbb{Q}$).
	\item If $\omega_{1},\ldots,\omega_{\phi(n)}$ are the $\phi(n)$ primitive nth roots of unity prove that any automorphism of $\mathbb{Q}(\omega)$ takes $\omega_{1}$ into some $\omega_{i}$.
	\item Prove that $[\mathbb{Q}(\omega):\mathbb{Q}]\leqslant\phi(n)$.
\end{enumerate}
\end{problem}

\begin{problem}\label{problem:14}
The notation is the same as that of Problem~\ref{problem:13}.
\begin{enumerate}
	\item Prove that there is an automorphism $\sigma_{i}$ of $\mathbb{Q}(\omega_{1})$ which takes $\omega_{1}$ into $\omega_{i}$;
	\item Prove that the polynomial $p_{n}(x)=(x-\omega_{1})(x-\omega_{2})\cdots(x-\omega_{\phi(n)})$ has rational coefficients. The polynomial $p_{n}(x)$ is called the nth cyclotomic polynomial.
	\item Prove that, in fact, the coefficients of $p_{n}(x)$ are integers.
\end{enumerate}
\end{problem}

\begin{problem}\label{problem:15}
Use the results of Problems~\ref{problem:13} and~\ref{problem:14} to prove that $p_{n}(x)$ is irreducible over $\mathbb{Q}$ for all $n\geqslant{1}$
\end{problem}
